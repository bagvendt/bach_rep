\documentclass[a4paper,12pt]{article}
\usepackage[utf8]{inputenc}
\usepackage{listings}
\usepackage{amsmath}    % need for subequations
\usepackage{graphicx}   % need for figures
\usepackage{verbatim}   % useful for program listings
\usepackage{color}      % use if color is used in text
\usepackage{subfigure}  % use for side-by-side figures
\usepackage{hyperref}   % use for hypertext links, including those to external documents and URLs
\usepackage{url}
\usepackage{tikz}
\usepackage{cancel}
\usepackage{algorithmic}
\usepackage{pgf}
\usetikzlibrary{arrows,automata}
\begin{document}
\title{Dokumentation af kode}
\author{Marcus Bjerg Gregersen\thanks{E-mail: marcusg@diku.dk} 
\and Claes Nøhr Ladefoged\thanks{E-mail: claes@diku.dk} }
\date{7. Maj 2011}
\maketitle
\newpage
\section*{Dokumentation}
Dokumentation er delt op i to dele. En python/Scipy/numpy del og en MATLAB del
\subsection*{Python}
Vores kode er kræver følgende:
\begin{itemize}
	\item Python 2.6.6
    \item Numpy 1.5.1
	\item Scipy 0.8.0
\end{itemize}
Vores python del består af følgende 4 filer
\texttt{proc.py}
\texttt{functions.py}
\texttt{constants.py}
\texttt{main.py}
I \texttt{functions.py} ligger alle vores funktioner, disse inkluderer for eksempel: foldning med en gausskerne og difference of gaussians.
\texttt{constants.py} indeholder forskellige konstanter og foldningskerner. I \texttt{proc.py} kan funktionerne kaldes med forskellige parametre, på et eller flere billeder, i den rækkefølge man ønsker.
En procedure har følgende syntaks:
\begin{verbatim}
PROCEDURE_NAVN = (BILLEDNAVN, [
                 (funktion1,
                   {eventuelle parametre til den enkelte funktion}),
                 (funktion2,
                   {parametre}),])
\end{verbatim}
I nedenstående kode giver vi et konkret eksempel på en procedure der indlæser et billede, bygger en gausskerne, og folder billedet med denne kerne, for til sidst at vise resultatet på skærmen:
\begin{figure}[H]
\begin{verbatim}
manuel_gauss_test = ( 
    'img/cirkel.png',
    [
    (manual_gauss,{
        'sigma':15,
        'x0':30,
        'y0':30
    }),
    (display,{})],
    )
\end{verbatim}
%\label{mis:sprog_manuel_gauss}
\end{figure}
Funktionerne bliver udført som forventet, i den rækkefølge de forekommer i listen.
Da vi i flere af de metoder vi benytter os af behøver de helt rigtige parametre for at opnå et brugbart resultat, har ovenstående syntaks vist sig at være meget nyttig da alle parametre ændres i samme fil.

Herefter tilføjes proceduren til en liste af flere andre procedurer. Denne liste pakkes ud og håndteres i \texttt{main.py}. Vores framework kaldes således: 
\begin{verbatim}
	python main.py <indeks i procedurelisten>
\end{verbatim}

Kaldet \texttt{'python main.py 0'} ville således kunne frembringe proceduren\\ \texttt{manuel\_gauss\_test}
\newpage
\subsection*{MATLAB}
Vores MATLAB kode er udviklet i MATLAB R2009b.


Vores MATLAB kode består af følgende 4 filer 
\begin{itemize}
	\item \texttt{euclidian.m}: Flytter en vesikeludklip med \texttt{nlfilter} over et vesikelbillede og udregner den euklidiske afstand. Mere information findes i vores rapport. 

	\item \texttt{SLD\_build.m}: Bygger et label- og intensitetsdictionary. 
	\item \texttt{SLD\_calc.m}: Udregner distancemappet som beskrevet i rapporten. 
	\item \texttt{SLD\_run.m}: Kalder \texttt{SLD\_calc.m} og thressholder resultatet.
\end{itemize}

	Fremgangsmåden når man ønsker at bruge SLD metoden
	er at bygge de to dictionaries med \texttt{SLD\_build.m}, for derefter at bruge dem i \texttt{SLD\_run.m}. Indtil videre hardkoder man selv hvilke billedfiler man ønsker at håndtere, thressholdværdierne hardkodes også.

	
\end{document}
