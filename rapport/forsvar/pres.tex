
\documentclass[12pt,t]{beamer}
\usepackage[utf8]{inputenc}
\usepackage{graphicx}
%\usepackage{amssymb,amsmath}
%\usepackage{verbatim} 
%\usepackage{url}
%\usepackage{float}
%\usepackage{color}
%\usepackage{listings}
%\usepackage{hyperref}
%\usepackage{subfig}
\usepackage{pslatex}
\usetheme[unit=ics,dk]{Frederiksberg}
\title{Detektion af vesikler i celler}
\subtitle{Bachelorforsvar}
\author{Claes Nøhr Ladefoged \and \\
Marcus Bjerg Gregersen}
\institute{Datalogisk Institut}
\date[]{20. Juni 2011}
\begin{document}
\frame[plain]{\titlepage}
\begin{frame}
\frametitle{Resultatet}
\begin{figure}[H]
	\centering
	\begin{minipage}[b]{0.49\linewidth}
		\includegraphics[scale=0.4]{img/ves/0.jpg}
	\end{minipage}
	%\hspace{0.5cm}
	\begin{minipage}[b]{0.49\linewidth}
		\includegraphics[scale=0.4]{img/ves/1_2.png}
	\end{minipage}
\end{figure}
\end{frame}

\begin{frame}
\frametitle{Introduktion}
\begin{figure}[H]
\includegraphics[scale=0.4]{img/orig/AN2-3_7.jpg}
\end{figure}
\end{frame}

\begin{frame}
\frametitle{Introduktion}
\begin{figure}[H]
\includegraphics[scale=0.4]{img/orig/AN2-3_7_new.jpg}
\end{figure}
\end{frame}

\begin{frame}
\frametitle{Billedbehandling - Fouriertransformation \& foldning}
Fouriertransformation
\begin{figure}[H]
\includegraphics[scale=0.35]{img/billedbeh/fourier.png}
\hspace{0.1cm}  for $x \in \mathbb{R}$
\end{figure}
Foldning
\begin{align*}
	f(t)*g(t)=\int_{-\infty}^{\infty}f(\tau)g(t-\tau)d\tau
\end{align*}
\end{frame}

\begin{frame}
\frametitle{Billedbehandling - Fouriertransformation}
\begin{figure}[H]
\includegraphics[scale=0.33]{img/billedbeh/lenaex.png}
\end{figure}
\end{frame}

\begin{frame}
\frametitle{Billedbehandling - Fouriertransformation}
\begin{figure}[H]
\includegraphics[scale=0.29]{img/billedbeh/lenaex3.png}
\end{figure}
\begin{figure}[H]
\includegraphics[scale=0.29]{img/billedbeh/lenaex1.png}
\end{figure}
\end{frame}

\begin{frame}
\frametitle{Billedbehandling - Foldning}
%Foldning opnås ved
\begin{align*}
	f(t)*g(t)=\int_{-\infty}^{\infty}f(\tau)g(t-\tau)d\tau
\end{align*}
%hvor $*$ er tegnet for foldning. 

\end{frame}

\begin{frame}
\frametitle{Billedbehandling - Gaussisk udglatning}
\begin{align*}
	G(x,y) = \frac{1}{2\pi\sigma^2}e^{-\frac{x^2+y^2}{2\sigma^2}}\label{ali:premethod_gaussG}
\end{align*}

\begin{figure}[H]
	\centering
	\includegraphics[scale=0.5]{../files/premethod/img/gausscell.png}
\end{figure}
\end{frame}

\begin{frame}
\frametitle{Billedbehandling - Sobelfilter}
\begin{align*}
	G_y = \begin{bmatrix}
		-1 & -2 & -1\\
		0 & 0 & 0\\
		1 & 2 & 1
	\end{bmatrix} * I
	&&
	G_x = \begin{bmatrix}
		-1 & 0 & 1\\
		-2 & 0 & 2\\
		-1 & 0 & 1
	\end{bmatrix} * I
	\end{align*}
\begin{align*}
	G &= \sqrt{G_x^2 + G_y^2}
	&\Theta = \arctan\left(\frac{G_y}{G_x}\right)
\end{align*}
\end{frame}

\begin{frame}
\frametitle{Billedbehandling - Sobelfilter}
\begin{figure}[H]
	\centering
	\includegraphics[scale=0.4]{../files/premethod/img/sobel3.png}
\end{figure}
\begin{figure}[H]
	\centering
	\includegraphics[scale=0.4]{../files/premethod/img/edgemap2.png}
\end{figure}
\end{frame}

\begin{frame}
\frametitle{Søgning med eksempelbilleder}
\begin{figure}[H]
	\centering
	\includegraphics[scale=0.4]{img/finalmethod/cell2.png}
\end{figure}
\end{frame}

\begin{frame}
\frametitle{Foldning med eksempelbilleder}
\begin{figure}[H]
	\centering
	\includegraphics[scale=0.4]{img/finalmethod/cell2.png}
\end{figure}

\begin{align*}
h(n_1,n_2) &= \sum_{k_1=-\infty}^{\infty} \sum_{k_2=-\infty}^{\infty} f(k_1,k_2)g(n_1-k_1,n_2-k_2)
\end{align*}

\end{frame}

\begin{frame}
\frametitle{Foldning med eksempelbilleder}

\begin{figure}[H]
	\centering
	\includegraphics[scale=0.6]{img/finalmethod/convolve.png}
\end{figure}
\end{frame}

\begin{frame}
\frametitle{Afstand til eksempelbilleder}
\begin{align*}
	||\vec{I}-\vec{J}||^2 = (\vec{I}-\vec{J})^T(\vec{I}-\vec{J}) = \vec{I}^T\vec{I} + \vec{J}^T\vec{J}-2\vec{I}^T\vec{J}
\end{align*}
\end{frame}

\begin{frame}
\frametitle{Afstand til eksempelbilleder}
\begin{align*}
	||\vec{I}-\vec{J}||^2 = (\vec{I}-\vec{J})^T(\vec{I}-\vec{J}) = \vec{I}^T\vec{I} + \vec{J}^T\vec{J}-2\vec{I}^T\vec{J}
\end{align*}

\begin{align*}
	\sum_m\sum_n \left(\frac{I_{ij}-mean(I_{ij})}{std(I_{ij})}-\frac{J-mean(J)}{std(J)}\right)^2
\end{align*}
\end{frame}


\begin{frame}
\frametitle{Segmentering med eksempelbilleder}
\begin{figure}[H]
\includegraphics[scale=0.35]{img/afstand/3.png}
\end{figure}
\end{frame}

\begin{frame}
\frametitle{Segmentering med eksempelbilleder}
\begin{figure}[H]
\includegraphics[scale=0.35]{img/afstand/4.png}
\end{figure}
\end{frame}

\begin{frame}
\frametitle{Segmentering med eksempelbilleder}
\begin{figure}[H]
\includegraphics[scale=0.35]{img/afstand/5.png}
\end{figure}
\end{frame}


\begin{frame}
\frametitle{Segmentering med eksempelbilleder}
\begin{figure}[H]
\includegraphics[scale=0.35]{img/afstand/6.png}
\end{figure}
\end{frame}


\begin{frame}
\frametitle{Segmentering med eksempelbilleder}
\begin{figure}[H]
\includegraphics[scale=0.35]{img/afstand/7.png}
\end{figure}
\end{frame}


\begin{frame}
\frametitle{Segmentering med eksempelbilleder}
\begin{figure}[H]
\includegraphics[scale=0.35]{img/afstand/8.png}
\end{figure}
\end{frame}

\begin{frame}
\frametitle{Segmentering med eksempelbilleder}
\begin{figure}[H]
\includegraphics[scale=0.35]{img/afstand/9.png}
\end{figure}
\end{frame}

\begin{frame}
\frametitle{Segmentering med eksempelbilleder}
\begin{figure}[H]
\includegraphics[scale=0.35]{img/afstand/10.png}
\end{figure}
\end{frame}

\begin{frame}
\frametitle{Segmentering med eksempelbilleder}
\begin{figure}[H]
\includegraphics[scale=0.35]{img/afstand/11.png}
\end{figure}
\end{frame}

\begin{frame}
\frametitle{Segmentering med eksempelbilleder}
\begin{figure}[H]
\includegraphics[scale=0.35]{img/afstand/12.png}
\end{figure}
\end{frame}

\begin{frame}
\frametitle{Segmentering med eksempelbilleder}
\begin{figure}[H]
\includegraphics[scale=0.35]{img/afstand/13.png}
\end{figure}
\end{frame}

\begin{frame}
\frametitle{Segmentering med eksempelbilleder}
\begin{figure}[H]
\includegraphics[scale=0.35]{img/afstand/14.png}
\end{figure}
\end{frame}


\begin{frame}
\frametitle{Segmentering med eksempelbilleder}
\begin{figure}[H]
\includegraphics[scale=0.35]{img/afstand/15.png}
\end{figure}
\end{frame}

\begin{frame}
\frametitle{Segmentering med eksempelbilleder}
\begin{figure}[H]
\includegraphics[scale=0.35]{img/afstand/16.png}
\end{figure}
\end{frame}

%ET kvarters præs, et kvarters disk
%Indledning krop afslutning. 
%Hav resultat på første slide
%Elevator speach
%Og så igang igen.


%Seperabel
%Foldning fungerer.

\begin{frame}
\frametitle{Resultater}
\begin{figure}[H]
	\centering
	\begin{minipage}[b]{0.49\linewidth}
		\includegraphics[scale=1]{../files/postmethod/img/eval_1-1.png}
	\end{minipage}
	%\hspace{0.5cm}
	\begin{minipage}[b]{0.49\linewidth}
		\includegraphics[scale=1]{../files/postmethod/img/eval_1-2.png}
	\end{minipage}
\end{figure}
\end{frame}


%Ves i et slide. med 1 og 2 <-- sensitivity (RECALL OG 1-PRES) 
%Recall graf
%Tal i et slide
%Kombineret res af 1-2 <--- MED RECALL OG 1-PRESS.
%TAL perspektiveret til tidligere tal??


\begin{frame}
\frametitle{SLUT}
Ikke som i slut men som i SLUT
\end{frame}

\end{document}
