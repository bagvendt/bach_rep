\section{Endelige metode}
\subsection{Metoderne}
\subsubsection{Sparse Label Dictionaries}	% Marcus
%\textbf{Disclaimer der måske skal med: SLD er en metode udviklet af Anders noget fra DTU, vi har set metoden til et foredrag men har ikke set noget litteratur der beskriver metoden udførligt. Dette er derfor vores udlægning af SLD-metoden}\\
%\\
%SLD er en metode til at klassificere forskellige elementer i et givent billede. I SLD vælger man et passende udsnit af kandidatbilledet hvorefter 


%Sparse Label Dictionaries-metoden (SLD) vælger man et passende udsnit af kandidatbilledet således at udsnittet kommer til at indeholde de forskellige klasser af elementer man ønsker at dedektere samt

Sparse Label Dictionaries metoden (SLD) har til formål at segmentere billeder. SLD opretholder to matricer med information om image patches, den ene indeholder intensiteten og den anden den assosierede label information.

SLD opbygger disse to matricer i to trin, første trin er at initialisere matricen. Dette gøres ved at se på to billeder, et træningsbillede samt et billede hvor ground truth er markeret. Fra starten vælges så en størrelse på det udsnit man ønsker at arbejde med. En atom klippes så ud af testbilledet i pixel (0,0) og med størrelse $\sqrt{n}\times\sqrt{n}\times h$, hvor $n$ er antallet af pixels i udsnittet og h er farvedybten, så f.eks. 1 ved grayscale og 3 ved RGB billeder. Denne atom gemmes så i intensitetsmatricen. I præcis samme område klippes en atom ud af ground truth billedet der så gemmes i labelmatricen. Ground truth billedet har også h dimensioner, hvor hvert segment der ønskes at detekteres er markeret men en farve. Vores udsnit flyttes så en pixel, hvorefter det samme gøres igen. Det vil sige at hvis testbilledet er $m\times m$ og vores atom størrelse er $n\times n$ så samler vi $(m-n+1)^2$ billeder. For at bygge de mest optimale matricer, vælges kun et tilfældigt subset af træningsbilledets udsnit. Ved en iterativ proces vælges så ...




\subsubsection{Foldning med eksempelbilleder} % Marcus
\subsubsection{Kombinationen} % Marcus
\subsection{Evaluering}
\subsubsection{ROC}
\subsubsection{Forbedringer af endelige metode}
\subsubsection{Hvad arbejder andre med} %
